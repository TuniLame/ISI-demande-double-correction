\documentclass{article}
\usepackage[utf8]{inputenc}
\usepackage[T1]{fontenc}
\usepackage{mathptmx}
\usepackage[scaled=.89]{helvet}
\usepackage{amsmath}
\usepackage{amssymb}
\usepackage[left=20mm,right=20mm,top=20mm,bottom=20mm]{geometry}
\usepackage{graphicx}
\setcounter{secnumdepth}{0}
\usepackage{hyperref}

\hypersetup{pdfborder={1 1 1}}
\def\LayoutTextField#1#2{\makebox[8em][l]{#1: }\raisebox{-.5ex}{#2}}
\def\LayoutChoiceField#1#2{\makebox[12em][l]{#1: }#2}
\renewcommand{\MakeTextField}[2]{{\vbox to #2{\vfill\hbox to #1{\hrulefill}}}}
\newdimen\longline
\longline=\textwidth\advance\longline-10em

\begin{document}
 \noindent\begin{Form}
\begin{flushleft}
 \includegraphics[width=4cm]{images/LogoISI.png}
\end{flushleft}
\vspace{-20mm}
\begin{flushright}

 %\TextField[name=nom,width=30mm,borderwidth=0,bordercolor={0 0 0},backgroundcolor=white,default=\today]{Tunis, le }
\TextField[name=date,width=25mm,borderwidth=0,bordercolor={0 0 0},backgroundcolor=white]{Tunis, le }
\end{flushright}
\vspace{16mm}
\begin{center}
 %\underline{\huge{Demande d'une double correction des copies d'examen}}
 \huge{Demande d'une double correction des copies d'examen}
 \end{center}
%\vspace{11mm}
%\underline{\Large{Partie réservée à l'étudiant}}
%\vspace{5mm}
\section{Partie réservée à l'étudiant}

\hspace{4.5mm}
\TextField[name=num_cin,width=40mm,borderwidth=0,bordercolor={0 0 0},backgroundcolor=white]{Numéro C.I.N.}\vskip1ex
\TextField[name=nom,width=\longline,borderwidth=0,bordercolor={0 0 0},backgroundcolor=white]{Nom}\vskip1ex
\TextField[name=prenom,width=\longline,borderwidth=0,bordercolor={0 0 0},backgroundcolor=white]{Prénom}\vskip1ex
\TextField[name=niv_etudes,width=86mm,borderwidth=0,bordercolor={0 0 0},backgroundcolor=white]{Niveau d'étude}
\hspace{10mm}
\TextField[name=groupe,width=15mm,borderwidth=0,bordercolor={0 0 0},backgroundcolor=white]{Groupe}\vskip1ex
\vspace{8mm}
\TextField[name=num_cin,width=86mm,borderwidth=0,bordercolor={0 0 0},backgroundcolor=white]{Matière}
\hspace{10mm}
\TextField[name=num_cin,width=15mm,borderwidth=0,bordercolor={0 0 0},backgroundcolor=white]{Salle}\vskip1ex


\end{Form}
\vspace{2mm}
\begin{flushright}
 Signature de l'étudiant
 
 \vspace{8mm}
 \dots\dots\dots\dots\dots\dots\dots
\end{flushright}
\vspace{-17mm}
\hspace{4.7mm}
\begin{tabular}{|c|}
\hline
 \begin{minipage}{121mm}
 \vspace{3mm}
 \hspace{3mm}
 \textbf{Important:} La nouvelle note (égale ou supérieure ou inférieur à l'ancienne note)
 
 \hspace{3mm} est comptabilisée dans le calcul de la moyenne de la matière.
 \vspace{3mm}
 \end{minipage}\\
 
 \hline
\end{tabular}


\vspace{12mm}
%\underline{\Large{Partie réservée à l'administration}}
%\vspace{5mm}
\section{Partie réservée à l'administration}

\hspace{4.5mm}
\begin{tabular}{|c|}
\hline
 \begin{minipage}{51mm}
 \vspace{3mm}
 \hspace{3mm}
 Code copie:
 \vspace{3mm}
 \end{minipage}\\
 
 \hline
\end{tabular}
\vspace{7mm}

\hspace{-2.6mm}
\begin{tabular}{ccc}
\begin{tabular}{|c|}
\hline
 \begin{minipage}{107mm}
 \vspace{3mm}
 \hspace{3mm}
 \underline{\large{Décision de l'enseignant correcteur}}
 \vspace{50mm}
 
 \hspace{3mm}
 \underline{\large{Nom et prénom et signature}}
 \vspace{20mm}
 \end{minipage}\\
 \hline
 \end{tabular}
 &
 &
 
 \begin{tabular}{|c|}
\hline
 \begin{minipage}{46mm}
 \vspace{3mm}
 \begin{center}
 \underline{\large{Avis du directeur de}}
 \underline{\large{département}}
 \end{center}
 \vspace{67mm}
 \end{minipage}\\
 \hline
 \end{tabular}\\


\end{tabular}
\end{document}